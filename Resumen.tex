\documentclass[11pt,a4paper]{article}
\usepackage[utf8]{inputenc} %para poder usar acentos 
\usepackage{amssymb}
\title{Resumen coloquio victor neumann lara 2020}
\date{\today}
\author{Rodrigo Chávez}

\newtheorem{theorem}{Teorema}
\newtheorem{lemma}{Lema}
\begin{document}
\maketitle
\section{Resumen}
Para un punto $x$ en el plano, una línea en forma de $L$ que consiste de un rayo vertical y uno horizontal que amanan de $x$ es llamada $L-linea$ con esquina en $x$. Sean $n$ puntos rojos y $n$ puntos azules en el plano lattice en posición general. Se sabe que el número de intersecciones al crear dos árboles geométricos de puntos rojos y azules en el plano depende de el número de alternancias de color en el cierre convexo. En el año 2013 Mikio Kano provó que se pueden construir 2 árboles, rojo y azul, con segmentos de $L-lineas$ con a lo más una intersección. Esta plática tiene como objetivo presentar los avances del problema donde se buscan dos trayectorias de segmentos de $L-lineas$, monocromáticas, ajenas que visiten cada punto haciendo uso de puntos de Steiner. Un punto Steiner es un punto que no es parte de la entrada original pero es agregado durante la solución del problema.
\section{Plática}
Primero que nada quisiera agradecer al comite organizador el haber aceptado mi platica, es la segunda edicion a la que asisito al coloqui y la primera en la que doy platica. Gracias.
Ahora, como ya vieron el titulo de mi platica consta de un sustantivo y 3 adjetivos, dejenme explicar esto primero introduciondo unas definiciones muy sencillas, las $L-lineas$, para un punto x en el plano una linea en forma de $L$ consitente de dos rayos uno vertical y uno horizontal emanentes de $x$ es llamado una $L-linea$ con esquina en $x$. Una linea vertical u horizontal que pasa por el punto $x$ son consideradas $L-lineas$ especiales. Para cada punto existen exactamente 6 $L-lineas$ con esquina en $x$ (hacer los dibujos en geogebra). Consideramos estas $L-lineas$ como lineas en el plano y consideramos problemas desde este punto de vista. Observe que para dos puntos no en la misma horizontal o vertical en entoces hay 2 $L-lineas$ que pasan a travez de estos dos puntos en comparacion de los puntos en el plano donde existe exactamente una linea que pasa por los puntos. Así pues existe una difenecia entre lineas y $L-lineas$.  Además podemos dar una definicion de posicion general por los medios de $L-linea$. Un conjunto $S$ de puntos se dice estar en posicion general si no tres puntos yacen en la misma $L-linea$ (poner dibujos que ejemplifiquen la diferencia entre las posiciones generales). Si $S$ esta en posicion general dada por la definicion anterior entoces el punto más alto (a la izquierda) y más bajo (a la derecha) de $S$ pueden estar en la misma vertical (horizontal). Sin embargo para cualquier otro punto $x \in S$ sí es cierto que se debe cumplir que ningún otro punto  en $S-\lbrace x \rbrace$ puede estar en la misma vertical vertical (horizontal). Prueba: observemos que si más de un punto yacen en la misma vertical y no son los extremos entonces puedo encontrar una $L-linea$ con los extremos de los lados, pero esto no puede pasar con los extremos, porque aunque esten en la misma vertical no pueden formar una $L-linea$ con ninguno de los demás puntos (poner dibujos para la prueba). Por lo tanto ambas definiciones difieren poco pero requieren las mismas condiciones para ambos para la mayoria de los puntos en $S$. 

Una vez dicho esto haré un pequeño parentesis para presentar un resultado de tokunaga para encontrar arboles generadores que no se intersecan (ajenos) que depende de la alternancia de colores en el cierre convexo de los puntos. 

Luego, para un conjunto $X$ de puntos en el plano llamamos $X-arbol$ a un arbol generador en $X$ cuyas aristas son segmentos de lineas que unen dos puntos en el conjunto.
\begin{theorem}
Sean $R$ y $B$ dos conjuntos ajenos de puntos rojos y azules tales que $R\cup B$ estan en posicion general, Sea $\tau\left( R, B \right)$ el numero de aristas $xy$ en el cierre convexo de $(R\cup B)$ tal que uno de $\lbrace x,y \rbrace$ es rojo y el otro es azul entonces el número de cruces en $T_{R} \cup T_{B}$ esta dado por $$max\left\lbrace \frac{\tau(R, B)}{2}, 0 \right\rbrace$$
\end{theorem}(voy intentar la pruba como la dice jorge y poner dibujos).
En particular podemos dibujar arboles ajenos si $\tau(R, B) \leq 2x|$. Voy a dar el sketch de la prueba, va como sigue. Primero encerramos todo en un triangulo que contenga 2 puntos de un color(sin perdida de generalidad digamos azules) y el tercero del otro(rojo), por el ahora digamos que agregamos 2 puntos para garantizar esta condicion auqnue no es necesario, omitiré los detalles de porqué esto es cierto ya que no es el resultado principal, ahora dentro de este triangulo si esta vacio o solo tiene puntos de color azul ya acabamos si no encontramos un punto de color rojo que me parte el triangulo en otros triangulo con la misma propieda de dos puntos de un color y el otro punto del otro color, procedemos por induccion y acabamos. Hay algunos detalles de como unir los puntos al arbol pero creanme que siempre se puede.
Este parentesis lo hice para ahora analogamente buscar arboles que no se intersecan utilizando segmentos de $L-lineas$,es decir, para un conjunto $X$ en el plano latice en posicion general, o sea no dos en la misma vertical u horizontal, podemos dibujar arboles generadores sin cruces en $X$ cuyas aristas son segmentos de $L-lineas$ que unen dos puntos en $X$, o sea un $X-arbol \, con \, segmentos \, de \, L-lineas$ . Aqui dos ejemplos de estos $X-arboles$ el segundo es un $X-arbol$ con  grado máximo 3. Estos van a ser importantes despues. Para un conjunto $S$ de puntos en al plano latice el cirre rectangular, denotado por $rect(S)$ es la caja rentangular más pequeña que encierre al conjunto $S$. Su cardinalidad puede ser 1 cuando solo hay un punto, 2 cuando la caja esta definida por las esquinas, 3 si esta definida por una esquina y 2 lados o 4 si esta definida por los 4 lados.
\begin{theorem}
Sean $R$ y $B$ conjuntos ajenos de puntos en el plano latice, $R \cup B$ en posicion general. Sea $\tau^*(R, B)$  el número de segmentos  $xy$ de $L-lineas$ en la frontera de $rect(R\cup B)$  tal que uno de $\{ x,y\}$ es rojo y el otro es azul. Entonces $\tau^*(R, B)$  es 0, 2 o 4. El maximo numero de cruces entre $R-tree$ y $B-tree$ es uno cuando $\tau^*(R, B) = 4$, Ademas sin $\tau^*(R, B) \leq 2$ podemos dibujar los arboles sin cruces con $\Delta(T_R) \leq 3$ y $\Delta(T_B) \leq 3$ 
\end{theorem}
Notemos que $\tau^*(R, B) = 4$ los arboles deben cruzarse al menos una vez. Digamos que $a,b \in R$ son los extemos verticales y $x,y \in B$ son los extremos horizontales entoces es imposible llegar de extremo a extremo sin atravezar una arista del otro arbol. Pero si agregamos un punto extra, al que llamamos de Steiner, un punto no del problema sin para ayudar a encontrar la solucion, entonces sí podemos unir los los extremos de un arbol para evitar la alternancia y el cruce.

Para provar el teorema 2 harán falta unas definiciones y un lema. Un poligono espiral ortogonal es un poligono cuyas fronteras consiste de dos cadenas de aristas llamdas interna y externa. Cada angulo interno de la cadena exterior es de $\frac{\pi }{2}$ y cada angulo externo de la cadena interna es de $\frac{3\pi }{2}$. observemos que pueden compartir aristas las cadenas entonces se veria como una arista.
\begin{lemma}
Sea $P$ un poligono espiral ortogonal en el plano latice y $S$ un conjunto de planos en posicion general contenido en $P$ y asumamos que cada arista de la cadena exterior tiene exactamente un punto y la cadena interior tambien tiene exactamente un punto o esta incluida en alguna cadena exterior. Entonces existe un $S-arbol \, T$  tal que $\Delta (T) \leq 3$ y $T$ esta dentro de $P$
\end{lemma}

Aqui la prueba: una arista interna incluida en una externa lo llamamos rectangulo aplanado. Notemos que un rectangulo aplanado puede contener un punto de $S$ como se puede observar en el dibujo. Ahi podemos econtrar dos rectangulos planos, uno con un punto y el otro sin puntos. Para preparar la construccion del $S-arbol$ descomponemos el poligono espiral ortogonal como se muestra en la figura donde $X, Y, Z$ denotan  rectangulos cerrados la arista superior de $Y$ esta contenida en la arista inferior de $X$ y la arista izquierdas de $X$ y $Y$ forman la arista de la cadena exterior de $P$ dos rectangulos consecutivos tienen las misma propiedades. Si $P$ no tiene rectangulos planos entonces los descomponemo como se muestra en la figura. Y si sí tiene rectangulos vacios los descomponemos como se muestra en esta otra figura donde algunos rectangulos son planos, esos los eliminamos entonces nos quedamos con algunos poligonos espiral sin rectangulos plano que podemos descomponer.

Observacion 1: no pueden existir 3 rectangulos vacios consecutivos.
Prueba: de otra manera alguna arista de la cadena exterior no tiene puntos lo que contradice la suposicion. 

Esta figura muestra el esquema de la construccion del $S-arbol$. Primero construiamos  una trayectoria del punto más "afuera" al de más adentro. y despues los unimos como se muestra en la segunda parte. Utilizando la cadena interna como referencia, digamos es el inferior, entonces podevos ver cual es el superior, izquierdo y derecho.
Las trayectorias que construimos en los rectangulos no planos $P_i$ tienen las siguientes propiedades \begin{enumerate}
\item comienzan en el punto superior y terminan el el inferior
\item $P_i$ pasa por todos los puntos del rectangulo
\item Cada segmento de $L-linea$ $xy$ tal que $x$ esta arriba de $y$ empieza en x hace derecha o izquierda y termina en y por arriba. Como en la figura.
\end{enumerate}
Para rectangulos planos con un punto $x_i$ este punto es el superior, inferior, derecho e izquierdo. Despues para rectangulos planos sin puntos pondremos un punto un punto dummy en el centro del rectangulo plano que tambien es el superior, inferior, derecho e izquierdo. En estos dos casos su trayectoria es un punto y vacia respectivamente

A continuacion conectamos trayecotorias $P_i$ y $P_{i+1}$ como sigue

Caso 1. $P_{i+1} \neq \emptyset $ 
En este caso siempre podremos conectar el inferior del rectangulo $i$ con el punto derecho del rectangulo $i+1$ como se ve en las siguientes figuras

Caso 2. $P_{i+1} =\emptyset $ 
En este caso siempre podremos conectar el punto de la derecha del rectangulo $i$ con el punto dummy del rectangulo $i+1$ con un segmento de $L-linea$ sin generar cruces. 

Consecuentemente ignorando los puntos dummy obtenemos un arbol $T$ en $S$ tal que \begin{enumerate}
\item $\Delta (T) \leq 3$
\item $T$ esta contenido en el poligono espiral
\end{enumerate}

Por la observacion 1 solo pueden haber siete casos, mostrados aqui pero los casos del 1 al 4 contadicen la suposicion de que cada arista de la cadena interna contiene exactamente un punto de $S$. En los demas casos las aristas son segmentos de $L-lineas$  $\hfill\ensuremath{\blacksquare} $

Ahora sí podemos probar el Teorema 2. Primero en el caso de que la solo haya dos alternancias de colores. i.e. $\tau^*(R, B) = 2$. Digamos que del rectangulo mas pequeño que contiene a $R \cup B$ el punto superior y el izquiero son rojos y el inferior y derecho son azules. Ahora toma el rectangulo $X_1 $más grande no contenga puntos azules cuya arista superior es la misma que la de $rect(R \cup B)$ y cuya arista inferior tenga un punto rojo y el rectangulo $Y_1$ más grande que no contenga puntos rojos, que su arista inferior sea la misma que la de $rect(R \cup B)$ y que contenga un punto azul en su arista superior. removamos esta seccion $X_1 \cup Y_1$ y lo que nos queda es rect$_2$ cuyo conjunto de puntos es $R_2$ y $B_2$. Ahora hacemos lo mismo para las caras derecha e izquierda de rect$_2$ y asi nos seguimos hasta que rect$(R_k\cup B_k)$ no contenga puntos  rojos ni azules. Ahora $X_1 \cup X_2 \ldots \cup X_k$ es una cadena espiral ortogonal que contiene los puntos rojos y $Y_1 \cup Y_2 \ldots \cup Y_k$  es una cadena espiral ortogonal que contiene los puntos azules. Claramente estas no se intersecan y por el lema 1 podemos encontrar un arbol de grado máximo 3 con segmentos de $L-lineas$. Ahora el caso en que $\tau^*(R, B) = 2$ pero en $rect(R\cup B)$ 3 puntos son rojos y uno es azul. La construccion es muy parecida solo que esta vez tomamos tres rectangulos en la primera iteracion para volver al caso anterior donde la en $rect(R\cup B)$ hay dos rojos y dos azules.

Ahora en caso donde $\tau^*(R, B) = 4$. Primero notemos que es imposible generar dos arboles que unan todos los puntos sin generar intersecciones porque si tenemos esta configuracion en la $rect(R\cup B)$ entonces no importa lo que pase los extremos rojos se tienen que unir, se parando a los azules. Entonces en nuestra construccion de los poligonos espirares podemos iniciar colocando una interseccion de la siguiente manera. Empezamos con el rectangulo que contenja rojos por la derecha, dejamos de considerar esos puntos, ahora una caja azul por abajo y de nuevo dejamos de considerar esos puntos. Obvsevemos que despues de eso ahora en los puntos restantes tenemos la alternancia del primer caso y por lo tanto a partir de ese punto podemos generar ambos poligonos espiral. Pero aun debemos conectar los primeros . esto lo hacemos atavesando el primer rectangulo de la espiral. Los putnos negros siempre tiene visibilidad entre ellos pero los blancos no, asi en un unico cruce unimos los rectangulos azules y los arboles quedan completos con un cruce que es inevitable.

Ahora si permitimos putnos de Steiner entonces en este unico caso donde se necesita un cruce para eliminarlo utilizamos un unico punto Steiner que elimina la alternancia  como se muestra en la imagen. 

Ahora vamos a extender este resultado buscando ya no arboles sino trayectorias, utilizando el menor número de puntos Steiner. 
Al construir las trayectorias en los rectangulos ya no tomaremos siempre el punto superior sino que partiremos en dos el rectangulo a partir del punto superior, (mostrar imagen) ahora la primer mitad la vamos a conectar no hacia un lado y hacia abajo, sino que iniciaremos por el punto más a la derecha y lo conectaremos siempre hacia la izquierda y hacia arriba o abajo segun haga falta. aqui un ejemplo,hasta llegar al punto que partio al rectangulo hacemos esto. El resto de los puntos los juntamos igual que como lo hicimos con los arboles. Hacemos esto porque en la constuccion de los arboles podiamos estar muy forzados en la direccion que tomaba las trayectorias de los rectangulos e imposibilitaba la conexion con el siguiente rectangulo (poner el ejmplo). Con esta construccion todo parece estar perfecto, miremos que si nuestro rectangulo tiene los suficientes puntos siempre podemos hacer cambios de direcciones en los rectangulo y nos da gran libertad al momento de unirlo, mas especificamente si tenemos al menos 4 puntos pueden suceder lo siguiente. Que esten de esta manera y la conectamos de esta manera (poner el ejmplo de 4 puntos) tambien pueden estar de esta manera y la conectamos asi o asi dependiendo de como querramos salir. y si la caja esta definida por 2 puntos es todavia mas facil, simplemente eliges como salir del rectangulo y el ultimo punto tiene total libertad. Así mismo con los rectangulos con 3 puntos podemos salir como aqui se ve. Por lo tanto ¿Podemos generar las dos treyecotiras sin cruces cuando la caja tiene solo 2 alternancias?(suspenso) Hasta ahora todo parece indicar que sí peeeeeeero, supongamos que los 3 puntos no estan así, sino que estan así. Con este pequéño cambio en la configuaracion, si le siguen rectangulos planos sin puntos no importa como le hagamos no vamos poder conectar los las trayectorias con $L-lineas$, de ahí dada esta configuracion podemos agregan un punto para tener una configuarcion de 4 puntos en la que sabemos que siempre podemos cambiar la orientacion de la trayectoria. Si esto es así entonces la pregunta ahora es cuantas veces se puede repetir esta configuracion que pareciera ser muy especifica, resulta que sí puede pasara muchas veces, una fraccion lineal de veces y aqui el ejemplo. Como podran observar por cada rectangulo con 3 puntos y 1 punto que define dos rectangulos planos hace falta un punto Steiner es decir $\frac{n}{4}$ puntos Steiner donde n es el número de puntos de un color. No obstante eso nos puede pasar en ambos poligonos es decir que nos haŕan $\frac{n}{2}$ en total. Lo cual suenan que son demasiados más porque creemos que se puede con un numero constante. y por qué creemos esto? hasta ahora no hemos encontrado un ejemplo que ojo de buen cubero requiera mas de un punto para generar ambas trayecorias sin intersecarse. Pues eso fue todo de mi parte. Gracias 

(esto es para despues: ahora teniendo estos arboles ajenos utilizando 1 punto steiner [decir que es un punto steiner] nos gustaria encontrar no arboles sino trayectorias, y es aquí donde extendimos el resultado dado por Mikio kano. Tambien pensar en poner que la conjetura se resulve parcialmente con el teorema que dice que se puede encajar cualquier arbol con grado maximo 3 estando todos estos en una trayetoria)
\\\\
Ahora $C_q[n,k,d]$ es un código lineal con $q$ símbolos, de longitud $n$, con dimensíon $k$, y distancia mínima $d$. 
Al ser de dimensión $k$ se tiene que el codigo tiene $2^{k}$ palabras de código.

\subsection{Suposiciones}
De este momento en adelante harán las siguientes suposiciones: 
\\$n \geq k$
\\$x_{1} = \mu_{1}, x_{2} = \mu_{2}, \cdots, x_{k} = \mu_{k}$ son los símbolos de infomación
\\$x_{k+1}, x_{k+2}, \cdots,x_{n}$ son los símbolos de paridad o de redundancia.
\\\\Ejemplo: Código de paridad par de longitud 4:
\begin{table}[h!]
\begin{center}
\caption{Código de paridad de longitud 4}
\label{tab:table1}
\begin{tabular}{l|l|l|r}
\textbf{$\mu_{1}$}&\textbf{$\mu_{2}$}&\textbf{$\mu_{3}$} \\
\hline
$x_{1}$ &$x_{2}$ &$x_{3}$ &$x_{4}$ \\
\hline
0 & 0 & 0 &0\\
0 & 0 & 1 &1\\
0 & 1 & 0 &1\\
0 & 1 & 1 &0\\
1 & 0 & 0 &1\\
1 & 0 & 1 &0\\
1 & 1 & 0 &0\\
1 & 1 & 1 &1\\
\end{tabular}
\end{center}
\end{table}

$x_4 = x_{1} + x_{2} + x_{3}$. Aquí nos referimos a la suma en el campo de los binarios
\end{document}
